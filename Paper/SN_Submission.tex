
% Template for Elsevier CRC journal article
% version 1.1 dated 16 March 2010

% This file (c) 2010 Elsevier Ltd.  Modifications may be freely made,
% provided the edited file is saved under a different name

% This file contains modifications for Procedia Computer Science
% but may easily be adapted to other journals

% Changes since version 1.0
% - elsarticle class option changed from 1p to 3p (to better reflect CRC layout)

%-----------------------------------------------------------------------------------

%% This template uses the elsarticle.cls document class and the extension package ecrc.sty
%% For full documentation on usage of elsarticle.cls, consult the documentation "elsdoc.pdf"
%% Further resources available at http://www.elsevier.com/latex

%-----------------------------------------------------------------------------------

%%%%%%%%%%%%%%%%%%%%%%%%%%%%%%%%%%%%%%%%%%%%%%
%%%%%%%%%%%%%%%%%%%%%%%%%%%%%%%%%%%%%%%%%%%%%%
%%                                          %%
%% Important note on usage                  %%
%% -----------------------                  %%
%% This file must be compiled with PDFLaTeX %%
%% Using standard LaTeX will not work!      %%
%%                                          %%
%%%%%%%%%%%%%%%%%%%%%%%%%%%%%%%%%%%%%%%%%%%%%%
%%%%%%%%%%%%%%%%%%%%%%%%%%%%%%%%%%%%%%%%%%%%%%

%% The '3p' and 'times' class options of elsarticle are used for Elsevier CRC
\documentclass[3p,times]{elsarticle}

%% The `ecrc' package must be called to make the CRC functionality available
\usepackage{ecrc}
\usepackage{setspace}
\usepackage{algcompatible}
\usepackage[usenames,dvipsnames]{color}
\usepackage{amsthm,amsmath}
\newtheorem{definition}{Definition}
\usepackage{graphicx}% Include figure files
\usepackage{dcolumn}% Align table columns on decimal point
\usepackage{bm}% bold math
\usepackage{hyperref}% add hypertext capabilities
\usepackage[mathlines]{lineno}

%% The ecrc package defines commands needed for running heads and logos.
%% For running heads, you can set the journal name, the volume, the starting page and the authors

%% set the volume if you know. Otherwise `00'
\volume{00}

%% set the starting page if not 1
\firstpage{1}

%% Give the name of the journal
\journalname{Social Networks}

%% Give the author list to appear in the running head
%% Example \runauth{C.V. Radhakrishnan et al.}
\runauth{}

%% The choice of journal logo is determined by the \jid and \jnltitlelogo commands.
%% A user-supplied logo with the name <\jid>logo.pdf will be inserted if present.
%% e.g. if \jid{yspmi} the system will look for a file yspmilogo.pdf
%% Otherwise the content of \jnltitlelogo will be set between horizontal lines as a default logo

%% Give the abbreviation of the Journal.
\jid{SN}

%% Give a short journal name for the dummy logo (if needed)
\jnltitlelogo{Social Networks}

%% Hereafter the template follows `elsarticle'.
%% For more details see the existing template files elsarticle-template-harv.tex and elsarticle-template-num.tex.

%% Elsevier CRC generally uses a numbered reference style
%% For this, the conventions of elsarticle-template-num.tex should be followed (included below)
%% If using BibTeX, use the style file elsarticle-num.bst

%% End of ecrc-specific commands
%%%%%%%%%%%%%%%%%%%%%%%%%%%%%%%%%%%%%%%%%%%%%%%%%%%%%%%%%%%%%%%%%%%%%%%%%%

%% The amssymb package provides various useful mathematical symbols
\usepackage{amssymb}
%% The amsthm package provides extended theorem environments
%% \usepackage{amsthm}

%% The lineno packages adds line numbers. Start line numbering with
%% \begin{linenumbers}, end it with \end{linenumbers}. Or switch it on
%% for the whole article with \linenumbers after \end{frontmatter}.
%% \usepackage{lineno}

%% natbib.sty is loaded by default. However, natbib options can be
%% provided with \biboptions{...} command. Following options are
%% valid:

%%   round  -  round parentheses are used (default)
%%   square -  square brackets are used   [option]
%%   curly  -  curly braces are used      {option}
%%   angle  -  angle brackets are used    <option>
%%   semicolon  -  multiple citations separated by semi-colon
%%   colon  - same as semicolon, an earlier confusion
%%   comma  -  separated by comma
%%   numbers-  selects numerical citations
%%   super  -  numerical citations as superscripts
%%   sort   -  sorts multiple citations according to order in ref. list
%%   sort&compress   -  like sort, but also compresses numerical citations
%%   compress - compresses without sorting
%%
%% \biboptions{comma,round}

% \biboptions{}

% if you have landscape tables
\usepackage[figuresright]{rotating}

% put your own definitions here:
%   \newcommand{\cZ}{\cal{Z}}
%   \newtheorem{def}{Definition}[section]
%   ...

% add words to TeX's hyphenation exception list
%\hyphenation{author another created financial paper re-commend-ed Post-Script}

% declarations for front matter

\begin{document}

\begin{frontmatter}

%% Title, authors and addresses

%% use the tnoteref command within \title for footnotes;
%% use the tnotetext command for the associated footnote;
%% use the fnref command within \author or \address for footnotes;
%% use the fntext command for the associated footnote;
%% use the corref command within \author for corresponding author footnotes;
%% use the cortext command for the associated footnote;
%% use the ead command for the email address,
%% and the form \ead[url] for the home page:
%%
%% \title{Title\tnoteref{label1}}
%% \tnotetext[label1]{}
%% \author{Name\corref{cor1}\fnref{label2}}
%% \ead{email address}
%% \ead[url]{home page}
%% \fntext[label2]{}
%% \cortext[cor1]{}
%% \address{Address\fnref{label3}}
%% \fntext[label3]{}

%\dochead{}
%% Use \dochead if there is an article header, e.g. \dochead{Short communication}

\title{Hierarchical Structure in Social Networks}

%% use optional labels to link authors explicitly to addresses:
%% \author[label1,label2]{<author name>}
%% \address[label1]{<address>}
%% \address[label2]{<address>}

\author[au1]{Cynthia Cook\footnote{Authors are listed in alphabetical order but all contributed equally to this publication.}} 
\author[au2]{Matthew J. Denny}
\author[au3]{Mitchell Goist} 
\author[au4]{Timmy Huynh}

\address[au1]{Department of Statistics cmc496@psu.edu}
\address[au2]{Department of Political Science, mzd5530@psu.edu}
\address[au1]{Department of Political Science mlg307@psu.edu}
\address[au2]{Department of Sociology and Criminology, tnh133@psu.edu}

\begin{abstract}

\end{abstract}

\begin{keyword}
Hierarchy \sep Network \sep Power

%% MSC codes here, in the form: \MSC code \sep code
%% or \MSC[2008] code \sep code (2000 is the default)

\end{keyword}

\end{frontmatter}

%%
%% Start line numbering here if you want
%%
% \linenumbers


\section{Introduction}
\label{sec:introduction}

\text{Hierarchies are imporantatskljdfl asdflks f.}

\begin{flushleft}
	Hierarchy is an important feature of many organizations, such as firms, social clubs, and military units. Formally, we can define a hierarchy as a system where people or groups are ranked according to status or authority. Yet it is difficult to operationalize this definition for measurement and comparison. There has been a great deal of research on power and status in groups and organizations, but most of this research relies on measurements defined over domain specific rankings, such as job titles. At the same time, networks scholars have defined a number of broadly applicable hierarchy metrics based on network structure, but these metrics are not necessarily grounded in meaningful sociological concepts of status and authority. Contrastingly, social theorists like Michael Mann have noted the messiness of society and that a network-oriented perspective of the ``sociospatial and organizational model [of a network]" can explicate the ``sources of social power," \cite{mann} but they have generally not delved into the methodologies through which to fully explore such power dynamics. In this paper, we seek to bring together these two areas of research, and to develop a framework for measuring hierarchy in social networks that is both generally applicable and exhibits a high degree of construct validity.
	
	\paragraph Having developed a framework for measuring hierarchy in networks, we will then test its internal and external validity. To test the internal validity of such a measure, we will conduct a simulation study. To test the external validity of our measure we will compare measurements across networks that we can theoretically rank by their degree of hierarchy. We will  then apply our framework to better understand the implications of hierarchical network structure for organizational performance in a sample of 17 county government organizations. 
\end{flushleft}

\subsection{Problem Statement}
\begin{flushleft}
	How do we define and measure hierarchy in (directed) social networks? We need to relate sociological conceptions of hierarchy and power to network measures. In developing this framework we intend to compare analytical and statistical approaches to measurement on both synthetic and real world datasets. In particular there are several key questions we must address.	
\end{flushleft}
\begin{enumerate}
	\item Is an analytical or statistical measure of network hierarchy more appropriate for our  goals?  
	\item Can we capture all or even most salient dimensions of hierarchy as defined in the sociological literature in a single measure?
	\item Can our measure be extended to undirected networks?
\end{enumerate}

\subsection{Significance}
\begin{flushleft}
	All of our fields including political science, sociology, and statistics approach the concept of hierarchy from different angles. Yet the main goal of any researcher is the same:  to accurately theoretically understand and quantify real world phenomena. Without statistical models/mathematical measurements for hierarchy which are theoretically based, and vice versa; theory that can be statistical/mathematically quantified and verified, the conceptual idea of hierarchy cannot be fully understood. We do not suggest that this project will achieve an overreaching theory and methods, but we strive to take the first step. At the very least, we will try to demonstrate the need for a united theory and corresponding methods. As an interdisciplinary team, we are in the unique position to accomplish our goals.
\end{flushleft}

\section{Measuring Hierarchy}

\subsection{Measuring Hierarchy in Groups}


\subsection{Measuring Hierarchy in Networks}


\section{A Model of Network Hierarchy}

\subsection{Notation}
Given a graph $G=(V,E)$, it is comprised of a set of vertexes $V=\{v_i\}={v_1,...v_N}$ and edges $E=\{e_j\}=e_1,...e_M$. We define the set of paths through the graph as $\{\pi_{k}\}=\pi_1,...\pi_P$. \\ Need to explain the collapsed graph for 3-5 in below...\\Need to define d in 6 \\need to know what m is in 7 \\
\subsection{Measures of Hierarchy}
\begin{enumerate}
	\item Landau's $h\in[0,1]$ is used to compare a directed network to a perfect linear hierarchy in \cite{animals}, where $S_i$ is the row sum for each node also referred to as the dominance total: 
	$$
	h=\frac{12}{N^3-N}\sum_{i=1}^{N}{[S_i-\frac{N-1}{2}]}
	$$
	\item Kendall's $K\in[0,1]$ is also used in \cite{animals} to compare a directed network to a perfect linear hierarchy. Let $d$ be the number of cyclic triad defined as: $d=\frac{N(N-1)(2N-1)}{12}-\frac{1}{2}\sum{S_i^2}$. Then: 
	$$
	K=1-\frac{d}{d_{max}},
	$$ 
	where $d_{max} = \left\{ \begin{array}{rl}
	\frac{1}{24}(N^3-N)&\mbox{ if $N$ is odd} \\
	\frac{1}{24}(N^3-4N)&\mbox{ if $N$ is even}
	\end{array} \right.$
	
	\item Treeness $T\in[-1,1]$ tries to capture how pyramidal the structure is and how unambiguous the chain of command is in the directed network. Mathematically, it is the average of $f(G)$ over the set in $W(G)$, where $W(G)$ is the subset $G_C$ and all of its subsets obtained through a leaf removal algorithm. This measure is defined in \cite{3D} as:
	$$
	f(G)=\frac{H_f(G_C)-H_b(G_C)}{max\{H_f(G_C),H_b(G_C)\}},
	$$
	where $H_f, H_b$ denote the forward and backward path entropies, respectively. Path entropy is defined to be:  $h_f(v_i)=-\sum{P(\pi_k|v_i)logP(\pi_k|v_i)}$.  
	
	\item Feedforwardness $F\in[0,1]$ tries to penalize cycles near the top of the directed network. Mathematically, it is the average of path weights $F(\pi_k)$ where cyclic modules that are closer to the top get a higher penalty, and defined by \cite{3D}. Here the paths under consideration consist of the subset starting from the top node of $G_C$. If there are $p$ such paths then:
	$$
	F(G)=\frac{1}{p}\sum_{k=1}^{p}{\frac{|v(\pi_k)|}{\sum{a_i}}},
	$$
	where $a_i$ are the weights of each node along the path under consideration (i.e. the number of collapsed nodes from $G$ in the corresponding node of $G_C$), and $v(\pi_k)$ is the number of nodes along the path $\pi_k$.
	
	\item Orderability $O\in[0,1]$ defines how orderable the directed network is. Mathematically, it is the fraction of nodes that do not belong to any cycle and defined by \cite{3D}:
	$$
	O(G)=\frac{|v_i\in V_c\cap V|}{|V|}
	$$
	
	\item Global Reaching Centrality, based on the $m$-reach centrality measure is adapted in \cite{GRC} to be a simple measure of hierarchy for any graph:
	$$
	GRC=\frac{\sum_{i\in V}{[C_R^{max}-C_R(i)]}}{N-1},
	$$
	\begin{enumerate}
		\item When the graph is unweighted and directed, the $C_R(i)$ is the local reaching centrality defined as the proportion of all nodes in $G$ that can be reached along outgoing edges from node $i$.
		
		\item When the graph is weighted and directed, the following version for the reaching centrality as defined in \cite{GRC} is used:
		$$
		C_{R}^{'}(i)=\frac{1}{N-1}\sum_{j: 0<d^{out}_{(i,j)<\infty}}{( \frac{\sum_{k=1}^{d^{out}(i,j)} {w_{i}^{(k)} (j) } }{d^{out}(i,j)} )}
		$$
		
		\item When the graph is unweighted and undirected, the following version for the reaching centrality as defined by \cite{GRC} is used:
		$$
		C_{R}^{''}(i)=\frac{1}{N-1}\sum_{j:0<d(i,j)<\infty}{\frac{1}{d(i,j)}}
		$$
	\end{enumerate}
	\item For social networks, \cite{online} defines hierarchy $h(G)\in[0,1]$ from inferred nodal rankings $r(v)$. Mathematically, it is defined as:
	$$
	h(G)=1-\frac{1}{m}A(G),
	$$ 
	where $A(G)=\sum_{(v_i,v_j)\in E} {max(r(v_i)-r(v_j)+1,0}$ is the total 'agony'. Since the rankings are not known, they are found by minimizing the total agony over all possible rankings $r$.
	
	\item Rooted Depth in directed networks is defined in \cite{depth}, where a root is a node that has only incoming edges. Given $r$ roots in a network, the each node has a local root depth equal to the average length of the shortest path between itself and all roots. Let $N_{r}$ be the number of node, root pairs in the network. Then the global root depth is defined as:
	$$
	D=\frac{1}{N_{r}}\sum_{i=1}^{N_r}{l_{ri}},	
	$$
	where $l$ is the length of the shortest path between root $r$ and node $i$.
	
	\item Relative Depth in directed networks is defined in \cite{depth} by first collapsing the network into $G_C$. Then the main root must be found, which lies on the end of the longest path through the network. From here, each vertex is given a relative depth $d$ equal to the path length between itself and the main root. Let $L=1,...l$ be the set of leaves and $R=1,...r$ be the set of roots. Then the global relative depth is found by:
	$$
	D=\frac{1}{l}\sum_{i=1}^{l}{d_i}-\frac{1}{r}\sum_{i=1}^{r}{d_i}
	$$
	
	\item Betweenness Centrality
	
	\item Eigenvector 
	
	\item Nodal Degree
	
	\item Closeness Centrality 
\section{Data}

\begin{flushleft}
	We are still working through evaluating a few different datasets to best suit our purposes. However, at present, this is a little difficult because we really want our measure to be theoretically-grounded, but we haven't yet developed a solid theoretical conception for hierarchy. Thus far, theory-wise, the Mann (1986) definition seems closest to the Liu-Driver measures discussed in the Mones et al. (2012) article: i.e., hierarchical networks are those in which the actions of a few nodes are needed to take control of the graph. Another potential definition, also implied, is hierarchy means the mechanisms of collective actions (i.e., the ability of different nodes to connect with one another) hinges on a small number.
	
	\paragraph Among the network datasets we are exploring, they are already or mostly in usable format. As we're navigating through our theoretical conception of hierarchy in network, we are ruling out the use of the karate club, dolphin, football, etc. datasets because we want to be able to analyze datasets where the networks are more interesting or theoretically-relevant. In this way, there are a couple of systems that might be useful. The first is a network of cooperation among militant groups, which encompasses joint exercises, mergers, and splits among militant groups: http://web.stanford.edu/group/mappingmilitants/cgi-bin/. This may interesting for us for a few reasons: (1) there is no de jure hierarchical structure (i.e., no formally-recognized chain of command or sovereignty); (2) militant groups face a classic collective action problem, and thus we can expect the dynamics Mann describes to hold; and (3) most theories of conflict would predict no hierarchy to occur in this system. An interesting system to compare this to would be military actions in Vietnam: http://tinyurl.com/pwofooy. The nodes here would be military units, and the edges are participation in the same battle. Of course, the main issue with this dataset is that it's undirected, which we've noted may be difficult to conceptualize within a hierarchy framework. We're focusing on conflict datasets because many of the theoretical definitions define hierarchy as essentially about outcomes—i.e., the ability of particular nodes to control the actions and behaviors of subordinate nodes.
	
	\paragraph We may also use manager network data where each organization has a ``county manager" who is theoretically in charge of the rest of the actors, providing an opportunity to determine if the methods we employ capture a plausible hierarchical structure. We are still exploring this and other datasets though.
\end{flushleft}


\begin{enumerate}

    \item Adolescent Health: survey asked students to list 5 male and female friends. \cite{AdHealth}
    
    \item Residence Hall: friendships between 217 students in Australian National University. \cite{Res}
    
    \item Taro Exchange: gift--giving relationships between households in a Papaun village. \cite{Taro}

    \item Highschool: friendship relationship between boys at a small Indiana high school in 1957-1958. \cite{HS}
    
    \item Dutch College: friendships between 32 university freshmen. \cite{Dutch}
    
    \item Monks: preference ratings between monks in a cloister during a crisis. \cite{monks}
    
    \item Physicians: innovation spread between 246 physicians in Illinois. \cite{docs}
    
    \item Seventh graders: activity specific proximity rankings for 29 middle school students in Victoria \cite{sevies}.
    
    \item Prosper loans: loans between users of prosper.com \cite{prosper}.
    
    \item Libimseti.cz: likes between users on a Czech dataing site \cite{libi}.
    
    \item Friendster: friendship adds on the online site Friendster \cite{friendster}.
    
    \item Digg: friendships on Digg \cite{digg}.
    
    \item Youtube: connections between Youtube users \cite{youtube}.
    
    \item Epinions: who--trusts--whom between users of epinions \cite{epinions}.
  
    \item EU emails: emails for 18 months from a major European research institution \cite{EU}.
    
    \item Facebook: friends lists from FAcebook, generated through a Facebook app survey \cite{facebook}.
    
    \item Google Plus: friends between users who selected to ``share circles'' on Google Plus \cite{facebook}.
    
    \item Linx kernel mailing list: communication network for the linux kernel mailing list, where each edge is a reply from a user to another \cite{linux}.

    \item Livejournal: map of an online community friendships of Livejournal users \cite{livejournal}.

    \item Manufacturing: communication network between employess of a mid--size manufacturing firm \cite{manufacturing}.
    
    \item Pokec: Friendship networks in the Pokec online social network, popular in Slovakia \cite{pokec}.
    
    \item Slashdot: tagging between users in slashdot for 2008 and 2009 \cite{livejournal}.
    
    \item Twitter: circles between twitter users \cite{facebook}.
    
    \item UC Irvine: messages sent between students on an online community at UC Irvine \cite{irvine}.
 
    \item U. Rovira i Virgili: email communication network from University Rovira i Virgili in Tarragona \cite{URV}.

    \item Wikipedia Talk: network of discussions between all users from the beginning of Wikipedia to January 2008 \cite{wiki}.

    \item Wikipedia Votes: data from administrator elections \cite{wiki}.
    
    \item Wikipedia Requests for Adminship: requests from 2003 through 2013 \cite{wiki2}.

    \item Friendster: network for online social site Friendster \cite{friendster}.

\end{enumerate}

\section{Analysis}
\begin{flushleft}
	The analytical portion of the problem will be conducted in R, which is known by all members of the group. We will be using both statistical and mathematical methods of quantifying and/or measuring hierarchy. We will focus on methods that have already been developed, published, and implemented in R, or are can easily be implemented by one of the group members. If time permits, we may try to develop or suggest directions for future development of our own statistical models and/or mathematical measurements. Each member of the group will be responsible for at least one method.
	
	\paragraph The statistical methods we will be looking into include hierarchical exponential graph models in the R package hergm. This package also includes hierarchical stochastic block models. Unlike fitting network data with exponential random graph models (ERGMs), hierarchical ERGMs focus on inducing local dependencies. Next, we will focus on latent space models, which can be fit in R using the latentnet package in the statnet suite of packages. For both the latent space and ERGM models, Bayesian inferential analysis can be conducted using the Bergm, VBLPCM, and lvm4net packages in R. We note that whenever fitting network data there is always the chance for computational timing and accuracy issues to come up. We have chosen a number of datasets for the purposes of capturing several types of hierarchies, but also so that we may have a few that are easily fit in R. Lastly, we will focus on mathematical measures of hierarchy. These measures primarily stem from graph theory, and can be easily programed by ourselves in R. The measurements include the Global Reach Centrality (GRC), Triangle Transitivity, Kendall's K, and Landau's lambda.
\end{flushleft}


\section{Conclusions}

\bibliographystyle{elsarticle-num}
\bibliography{library}

%% Authors are advised to use a BibTeX database file for their reference list.
%% The provided style file elsarticle-num.bst formats references in the required Procedia style

%% For references without a BibTeX database:

% \begin{thebibliography}{00}
\begin{thebibliography}{9}
	\bibitem{animals} 
	Shizuka, D. and McDonald, D. B. (2012),
	\textit{A social network perspective on measurements of dominance hierarchies}. 
	Animal Behavior: 83, 925-934.
	
	\bibitem{3D} 
	Murtra, B., Goni, J., and Caso, C. (2013),
	\textit{On the origins of hierarchy in complex networks}. 
	PNAS: 110, 33, 13316-13321.

	\bibitem{GRC} 
	Mones, E., Vicsek, L., and Vicsek, T. (2012),
	\textit{Hierarchy Measure for Complex Networks}. 
	Plos ONE: 7, 3, 1-10.
	
		\bibitem{online} 
		Gupte, M., Shankar, P., Li, J., Muthukrishnan, S., and Iftode, L. (2011)
		\textit{Finding Hierarchy in Directed Online Social Networks}. 
		International World Wide Web Conference Committee (IW3C2).
		
		\bibitem{depth} 
		Suchecki, K. and Holyst, J. (2013),
		\textit{Hierarchy depth in directed networks}. 
		Physica A: preprint.
		
	\bibitem{Liu12}
	Liu, Y., Slotine, J., and Barab{\'a}si, A (2012).
	\textit{Control centrality and hierarchical structure in complex networks}.
	Plos ONE: 7, 9.
	
	\bibitem{AdHealth}
	Moody, J. (2001).
	\textit{Peer influence groups: Identifying dense clusters in large networks}.
	Social Networks: 23, 4, 261-283.
	
	\bibitem{Res}
	Freeman, L., Webster, C., Kirke, D. (1998)
	\textit{Exploring social structure using dynamic three--dimensional color images.}
	Social Networks: 20, 2, 109-118.
	
	\bibitem{Taro}
	Schwimmer, E. (1973)
	\textit{Exchange in the Social Structure of the Orokaiva: Traditional and Emergent Ideologies in the Northern District of Papua}.
	St. Martin's Press.
	
	\bibitem{HS}
	Coleman, J. (1973)
	\textit{Introduction to mathematical sociology}.
	London Free Press Glencoe.
	
	\bibitem{Dutch}
	Van de Bunt, G., Van Deuijn, M., Snijders, T. (1999)
	\textit{Friendship networks through time: An actor-oriented dynamic statistical network model}.
	Computational and Mathematical Organization Theory: 5, 2, 167-192.
	
	\bibitem{monks}
	Breiger, R., Boorman, S., Arabie, P. (1975)
	\textit{An algorithm for clustering relational data with applications to social network analysis and comparison with multidimensional scaling}.
	Journal of Mathematical Psychology: 12, 3, 1975.
	
	\bibitem{docs}
	Coleman, J., Katz, E., Menzel, H. (1957)
	\textit{The diffusion of an innovation among physicians}.
	Sociometry: 253-270.
	
	\bibitem{sevies}
	Watts, D., Strogatz, S. (1998)
	\textit{Collective dynamics of `small world' networks}.
	Nature: 393, 1, 440-442
	
	\bibitem{cite}
	Prosper loans network dataset - KONECT, May 2015.
	
	\bibitem{libi}
	Brozovsky, L., Petricek, V. (2007)
	\textit{Recommender system for online dating service}.
	Proc. Znalosti: 29-40.
	
	\bibitem{friendster}
	Friendster network dataset - KONECT, May 2015.
	
	\bibitem{Digg}
	Hogg, T., Lerman, K. (2012)
	\textit{Social dynamics of Digg}.
	EPJ Data Science: 1, 5.
	
	\bibitem{youtube}
	Mislove, A., Marcon, M., Gummadi, K., Druschel, P., Bhattacharjee, B. (2007)
	\textit{Measurement and analysis of online social networks}.
	Proc. Internet Measurement Conference.
	
	\bibitem{epinions}
	Richardson, M., Agrawal, R., Domingos, P. (2003)
	\textit{Trust Management for the Semantic Web}.
	ISWC
	
	\bibitem{EU}
	Leskovec, J., Kleinber, J., Faloutsos, C. (2007)
	\textit{Graph Evolution: Densification and Shrinking Diameters}.
	ACM Transactions on Knowledge Discovery from Data (ACM TKDD): 1, 1.
	
	\bibitem{facebook}
	McAuley, J., Leskovec, J. (2012)
	\textit{Learning to Discover Social Circles in Ego Networks}.
	NIPS
	
	\bibitem{linux}
	Linux kernel mailing list replies network dataset - KONECT, May 2015.
	
	\bibitem{livejournal}
	Leskovec, J., Lang, K., Dasgupta, A., Mahoney, M. (2009)
	\textit{Community Structure in Large Networks: Natural Cluster Sizes and the Absence of Large Well-Defined Clusters}.
	Internet Mathematics: 6, 1, 29--123.
	
	\bibitem{manufacturing}
	Michalski, R., Palus, S., Kazienko, P. (2011)
	\textit{Matching organizational structure and social network extracted from email communication}.
	Lecture Notes in Business Information Processing: 87, 196-206.
	
	\bibitem{pokec}
	Takac, L., Zabovsky, M. (2012)
	\textit{Data Analysis in Public Social Networks}.
	International Scientific Conference \& International Workshop Present Day Trends of Innovations, Lomza, Poland.
	
	\bibitem{irvine}
	Opsahl, T., Panzarasa, P. (2009)
	\textit{Clustering in weighted networks}.
	Social Networks: 31, 2, 155-163.
	
	\bibitem{URV}
	Guimera, R., Danon, L., Diaz-Guilera, A., Giralt, F., Arenas, A. (2003)
	\textit{Self--similar community structure in a network of human interactions}.
	Phys. Rev. E.: 68, 6.
	
	\bibitem{wiki}
	Leskovec, J., Huttenlocher, D., Kleinberg, J. (2010)
	\textit{Predicting Positive and Negative Links in Online Social Networks}.
	WWWW
	
	\bibitem{wiki2}
	West, R., Paskov, H., Leskovec, J., Potts, C. (2014)
	\textit{Exploiting Social Network Structure for Person-to-Person Sentiment Analysis}.
	Transactions of the Association for Computational Linguistics: 2, 297-310.

    \bibitem{friendster}
    Friendster network dataset - KONECT, May 2015.
    
    \bibitem{mann}
    Mann, Michael. ``Chapter 1: Societies as organized power networks."
    \textit{The Sources of Social Power, Volume 1, A history of power from the beginning to AD 1760}. Cambridge University Press, 1986: 1-33.
    
\end{thebibliography}
%% \bibitem must have the following form:
%%   \bibitem{key}...
%%

% \bibitem{}

% \end{thebibliography}
\end{enumerate}
\end{document}

%%
%% End of file `ecrc-template.tex'. 
